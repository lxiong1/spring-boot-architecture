\section*{Support for Future Releases of Java}

\begin{textbox}
	\obeylines
	\textbf{Change Case:} Support compatibility with future releases of Java.
	\textbf{Likelihood:} It's nearly guaranteed that Spring Boot will support future releases of Java since the Spring ecosystem is built on the JVM.
	\textbf{Impact:} The impact of upgrading the Spring Boot system to support new releases of Java will initially be small, but will increase incrementally as more versions are released.
\end{textbox}

\medskip

Upgrading to the latest release of fundamental components, like a programming language, is a significant undertaking, especially for legacy applications that require a lot of maintenance. In the case of Spring Boot, the upgrade is inevitable because the Spring ecosystem is built to run on the JVM. The framework's usefulness will die without support for future Java version releases.

According to a 2018 Snyk survey, 79 percent of Java applications in production run on Java SE 8 \cite{snyk:online}. However, 7 in 10 of those applications used the Oracle JDK \cite{snyk:online}. Oracle announced that Java SE 8 public updates for businesses would end in January 2019, and require a commercial license for extended support \cite{oraclejava:online}. This means that most organizations will be forced to migrate to a more current version of Java and that Spring Boot will have to accommodate the Java version end-of-life cycle.