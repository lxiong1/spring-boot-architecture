\section*{Support GraphQL for Web Service Development}

\begin{textbox}
	\obeylines
	\textbf{Change Case:} Provide official support for GraphQL implementations.
	\textbf{Likelihood:} It is unlikely that Spring Boot will support GraphQL implementations since most companies have yet to discover what it is, how to use it, and how it can be beneficial.
	\textbf{Impact:} Supporting GraphQL implementations would make a significant impact since Spring Boot is so widely used, and as data becomes increasingly important to businesses.
\end{textbox}

\medskip

At a high level, GraphQL, like REST, is simply a request-and-response transaction. The main difference is that GraphQL allows for the flexibility to make exact queries, whereas, with REST, it may take multiple requests to receive the desired data. This is one reason data behemoths like GitHub, Twitter, and Pinterest use GraphQL for data distribution.

Spring Boot is primarily concerned with API development using the REST principles. Many companies continue to benefit from this architectural style, but as the global market becomes more dependent on data, GraphQL may become a more efficient solution.