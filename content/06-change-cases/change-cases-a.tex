\section*{Kotlin Documentation}

\begin{textbox}
	\obeylines
	\textbf{Change Case:} Add Kotlin code-snippets to Spring Boot documentation.
    
    \textbf{Likelihood:} It is likely that the Spring Boot documentation will soon include Kotlin code-snippets since the application now includes support for Kotlin extensions.
    
    \textbf{Impact:} The impact of including Kotlin code-snippets in the Spring Boot documentation is unknown, but it has the potential to be significant if Kotlin continues to become more popular.
\end{textbox}

\medskip

Kotlin is a programming language that was first released in July 2011. It was built to be a more feature-rich, precise, and type-safe version of Java with a similarly fast compile time \cite{kotlinwiki:online}. Kotlin runs on the Java Virtual Machine (JVM), which gives it the advantage of being completely interoperable with Java. It is roughly estimated to require 40 percent fewer lines of code than that of its Java-equivalent \cite{kotlinjetbrains:online}.

Kotlin's first stable release was in February 2016. Each year since then, developer interest in the language has risen, according to the developer-centric research organization HackerRank.

The Spring Boot team has slowly started developing features using Kotlin, furthering the support of Kotlin-compatible code. Evidence of this can be found with a search for \texttt{.kt} files in the Spring Boot repository. The command \texttt{find . -name '*.kt' | xargs wc -l} returns a count of 1044 total lines of Kotlin code in the Spring Boot codebase. Although this may seem insignificant compared to the nearly 100,000 lines of code written in Java, the Spring Boot team has introduced the language into the Spring ecosystem, which likely means Kotlin is here to stay.