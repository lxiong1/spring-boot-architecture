\section{Stakeholders}
\subsection*{Acquirers}
Acquirers are decision-makers. Rod Johnson, who started Spring Boot, may be considered its original acquirer. Acquirers are typically concerned with the cost, plan, and return-on-investment; however, since Spring Boot is an open-source framework that is implemented by other projects, the most relevant acquirers, in this case, are the technical directors that approve the implementation of Spring Boot in the project that they oversee.

\subsection*{Assessors}
Assessors are concerned with standards, quality, and compliance with laws and regulations. Maintainers of Spring Boot have implemented automated assessors (Git Hooks and Checkstyles) to ensure that the open-source project maintains a high level of quality. Spring Boot does not, however, have to conform to any legal standards, even if the applications that implement Spring Boot may be regulated.

\subsection*{Communicators}
Communicators are concerned with effectively conveying the purpose and value of a product to its stakeholders. Communicators must have a deep understanding of the system so that they can document it well. Spring Boot documentation is extensive and well maintained by Pivotal, the company that supports the Spring ecosystem, and the project's contributors.

\subsection*{Developers} 
Developers build and deploy software. Spring Boot is a popular open-source project that is maintained by contributors from around the world.

\subsection*{Maintainers}
Maintainers are liable for managing the evolution of the system. Maintainers make sure that the Communicators are up to date on the release documentation. Pivotal developers and Spring Boot contributors are its maintainers.

\subsection*{Production Engineers}
Production engineers are responsible for the management of deployed systems. Because Spring Boot is a framework that is implemented by deployable applications and isn't itself deployed, the project doesn't have production engineers.

\subsection*{Suppliers}
Suppliers provide the third-party software, hardware, and infrastructure required to build, deploy, and maintain a software product. Although Spring Boot is not a deployable application, it does run on the Java Virtual Machine (JVM) and is therefore dependent on the Java Development Kit (JDK). It follows that the Oracle Corporation, which maintains Java, can be considered a supplier to Spring Boot.

\subsection*{Support Staff}
Support Staff are concerned with providing user support. Although Spring Boot is well documented, and there is an extensive community of users that answer questions related to the framework online, the project doesn't have any official Support Staff.

\subsection*{System Administrators}
System Administrators are in charge of monitoring and management of the system once it has been deployed to production. Although Spring Boot applications need System Administrators, the Spring Boot project does not.

\subsection*{Testers}
Testers are responsible for testing the system to verify the correctness of the product and its specifications. Spring Boot contributors are responsible for testing their code. Spring Boot does not seem to have a dedicated group of testers.

\subsection*{Users}
Users are in charge of defining the functional requirements of a system. Spring Boot users are Java developers that build web services.