\section{Architectural Principles}

\noindent\makebox[\linewidth]{\rule{\textwidth}{3pt}}\ \\

\textbf{Principle}: The Spring Boot framework when used within a consuming project must greatly minimize the effort of writing configurations for developing a Spring-based web application.\\

\textbf{Rationale}: If Spring Boot does not reduce the effort for Java developers to write boilerplate configuration code, it will not have a convincing reason to be adopted over Spring Framework or other web application frameworks.\\

\textbf{Implications}: 
\begin{itemize}
\item Spring Boot must use Spring Framework as its' foundation.
\item Spring Boot must be written in Java or other JVM-based programming languages.
\item Spring Boot must abstract away all configuration code into a simple interface from the perspective of a Java developer.
\item A consuming project must be able to use short, expressive, and single-lined annotation abstractions to auto-configure all class path dependencies.
\item A consuming project must have a way to pre-define objects to be configured by code that is understood by Spring Boot.
\item Spring Boot features and documentation must be up to date with software technology trends to stay relevant to Java developers.
\end{itemize}

\noindent\makebox[\linewidth]{\rule{\textwidth}{3pt}}\ \\

\textbf{Principle}: The Spring Boot framework when used within a consuming project must be able to delivery value to the wide-ranged use cases of Java developers within the context of web application development.\\

\textbf{Rationale}: Java Developers all have unique situations, circumstances, and environments of which they work in. There is no one-size-fits-all type of framework so Spring Boot has to justify their flexibility and adaptability to tend to the needs of Java developers.\\

\textbf{Implications}: 
\begin{itemize}
\item Spring Boot must be able to integrate with a wide range of external libraries concerning web application development.
\item Spring Boot must be able to acknowledge and address the concerns of integration with other libraries.
\item Spring Boot must package commonly used and cohesive libraries together to address a particular need.
\item Spring Boot must be easily accessible to Java developers through a repository.
\end{itemize}

\noindent\makebox[\linewidth]{\rule{\textwidth}{3pt}}\ \\

\textbf{Principle}: The Spring Boot framework must be developed by reusing many functionalities of the Spring Framework.\\

\textbf{Rationale}: Code reuse is a large part of developing software to reduce redundancy and shorten development time.\\

\textbf{Implications}: 
\begin{itemize}
\item Spring Boot must adhere to some the design constraints of Spring Framework.
\item Spring Boot must work in harmony with Spring Framework and not against it.
\item Spring Boot must be able to market itself apart from Spring Framework.
\item Spring Boot must offer something to Java developers that Spring Framework already does not.
\end{itemize}

\noindent\makebox[\linewidth]{\rule{\textwidth}{3pt}}
